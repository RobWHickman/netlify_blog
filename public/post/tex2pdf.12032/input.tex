\documentclass[]{article}
\usepackage{lmodern}
\usepackage{amssymb,amsmath}
\usepackage{ifxetex,ifluatex}
\usepackage{fixltx2e} % provides \textsubscript
\ifnum 0\ifxetex 1\fi\ifluatex 1\fi=0 % if pdftex
  \usepackage[T1]{fontenc}
  \usepackage[utf8]{inputenc}
\else % if luatex or xelatex
  \ifxetex
    \usepackage{mathspec}
  \else
    \usepackage{fontspec}
  \fi
  \defaultfontfeatures{Ligatures=TeX,Scale=MatchLowercase}
\fi
% use upquote if available, for straight quotes in verbatim environments
\IfFileExists{upquote.sty}{\usepackage{upquote}}{}
% use microtype if available
\IfFileExists{microtype.sty}{%
\usepackage{microtype}
\UseMicrotypeSet[protrusion]{basicmath} % disable protrusion for tt fonts
}{}
\usepackage[margin=1in]{geometry}
\usepackage{hyperref}
\hypersetup{unicode=true,
            pdftitle={Riddler 27th April 2018},
            pdfauthor={Robert Hickman},
            pdfborder={0 0 0},
            breaklinks=true}
\urlstyle{same}  % don't use monospace font for urls
\usepackage{graphicx,grffile}
\makeatletter
\def\maxwidth{\ifdim\Gin@nat@width>\linewidth\linewidth\else\Gin@nat@width\fi}
\def\maxheight{\ifdim\Gin@nat@height>\textheight\textheight\else\Gin@nat@height\fi}
\makeatother
% Scale images if necessary, so that they will not overflow the page
% margins by default, and it is still possible to overwrite the defaults
% using explicit options in \includegraphics[width, height, ...]{}
\setkeys{Gin}{width=\maxwidth,height=\maxheight,keepaspectratio}
\IfFileExists{parskip.sty}{%
\usepackage{parskip}
}{% else
\setlength{\parindent}{0pt}
\setlength{\parskip}{6pt plus 2pt minus 1pt}
}
\setlength{\emergencystretch}{3em}  % prevent overfull lines
\providecommand{\tightlist}{%
  \setlength{\itemsep}{0pt}\setlength{\parskip}{0pt}}
\setcounter{secnumdepth}{0}
% Redefines (sub)paragraphs to behave more like sections
\ifx\paragraph\undefined\else
\let\oldparagraph\paragraph
\renewcommand{\paragraph}[1]{\oldparagraph{#1}\mbox{}}
\fi
\ifx\subparagraph\undefined\else
\let\oldsubparagraph\subparagraph
\renewcommand{\subparagraph}[1]{\oldsubparagraph{#1}\mbox{}}
\fi

%%% Use protect on footnotes to avoid problems with footnotes in titles
\let\rmarkdownfootnote\footnote%
\def\footnote{\protect\rmarkdownfootnote}

%%% Change title format to be more compact
\usepackage{titling}

% Create subtitle command for use in maketitle
\newcommand{\subtitle}[1]{
  \posttitle{
    \begin{center}\large#1\end{center}
    }
}

\setlength{\droptitle}{-2em}
  \title{Riddler 27th April 2018}
  \pretitle{\vspace{\droptitle}\centering\huge}
  \posttitle{\par}
  \author{Robert Hickman}
  \preauthor{\centering\large\emph}
  \postauthor{\par}
  \predate{\centering\large\emph}
  \postdate{\par}
  \date{2018-05-01}


\begin{document}
\maketitle

I've been looking for small programming problems to practice on while
running experiments. One such source is
\href{https://fivethirtyeight.com/features/how-fast-can-you-type-a-million-letters/}{Fivethirtyeight's
Riddler} column which posts conundrums weekly. This week one problem
focus on one of life's universal problems: how many urinals are needed
in any bathroom for all patrons to use it without awkwardness.

Formally this is phrased as:

\begin{quote}
Some number, N, of people need to pee, and there is some number, M, of
urinals in a row in a men's room. The people always follow a rule for
which urinal they select: The first person goes to one on either far end
of the row, and the rest try to maximize the number of urinals between
them and any other person. So the second person will go on the other far
end, the third person in the middle, and so on. They continue to occupy
the urinals until one person would have to go directly next to another
person, at which point that person decides not to go at all.
\end{quote}

\begin{quote}
What's the minimum number, M, of urinals required to accommodate all the
N people at the same time?
\end{quote}

Which is perhaps easiest explained using the `urinal etiquette' meme:

\emph{}

Luckily, this sort of problem is extremely tractable in R to get an
estimate of the function for any 1:N people with a few simple loops:

```\{r, warning=FALSE,message=FALSE\} \#just going to use dplyr and purr
\#data.table might be faster but not too worried- verbose programming
anyway library(dplyr) library(purrr)

\section{a tip from colin fay}\label{a-tip-from-colin-fay}

\section{\texorpdfstring{\url{https://twitter.com/_ColinFay/status/987260548344631298?ref_src=twsrc\%5Etfw\&ref_url=https\%3A\%2F\%2Frweekly.org\%2F}}{https://twitter.com/\_ColinFay/status/987260548344631298?ref\_src=twsrc\%5Etfw\&ref\_url=https\%3A\%2F\%2Frweekly.org\%2F}}\label{httpstwitter.com_colinfaystatus987260548344631298ref_srctwsrc5etfwref_urlhttps3a2f2frweekly.org2f}

\texttt{\%not\_in\%} \textless{}- negate(\texttt{\%in\%})

\section{start with n = 1 and with a bathroom with 1
urinal}\label{start-with-n-1-and-with-a-bathroom-with-1-urinal}

n \textless{}- 1 urinal\_number \textless{}- 1

\section{create a df with 1 urinal which is
unoccupied}\label{create-a-df-with-1-urinal-which-is-unoccupied}

urinals\_df \textless{}- data.frame(urinal = 1:urinal\_number, occupied
= rep(NA, urinal\_number))

\section{for how many n do we want to
solve}\label{for-how-many-n-do-we-want-to-solve}

while(n \textless{} 10) \{ \#whilst not all n have a urinal to use loop
through
while(sum(urinals\_df\(occupied, na.rm = TRUE) < n) {  #when all are unoccupied take the first urinal  if(sum(urinals_df\)occupied,
na.rm = TRUE) == 0) \{
urinals\_df\(occupied[1] <- 1  #when all but 1 are unoccupied and there are more than 2 urinals take the opposite end one next  } else if(sum(urinals_df\)occupied,
na.rm = TRUE) == 1 \& nrow(urinals\_df) \textgreater{} 2) \{
urinals\_df\(occupied[nrow(urinals_df)] <- 1  #otherwise work out the most isolated free urinal  } else {  #get the distances from each urinal to all the occupied urinals  urinal_distances <- abs(1:nrow(urinals_df) - rep(which(!is.na(urinals_df\)occupied)),
each = nrow(urinals\_df))) \%\textgreater{}\% matrix(., nrow =
length(!is.na(urinals\_df\$occupied))) \#index
rownames(urinal\_distances) \textless{}- 1:nrow(urinal\_distances)

\begin{verbatim}
  #awkward urinals are ones that are either taken or next to taken urinals
  #don't want to urinate there
  awkward <- c(which(urinal_distances == 1, arr.ind = TRUE)[,1], 
               which(urinal_distances == 0, arr.ind = TRUE)[,1]) %>%
    unique()
  
  #use %not_in% to find free urinals that aren't in an awkward position
  possible_urinals <- which(rownames(urinal_distances) %not_in% awkward)
  
  #if only one remains use this urinal
  if(length(possible_urinals) == 1) {
    taken_urinal <- possible_urinals
  } else if(length(possible_urinals) > 1) {
    #for the remaining possible urinals find how far the closest taken urinal is
    #initialise a small nameless func
    closest_distance <- lapply(seq(nrow(urinal_distances)), function(x){
      row <- urinal_distances[x,]
      min <- min(row)
      }) %>%
      unlist()
    
    #use the urinal that has the maximum distance to its closest urinal
    taken_urinal <- as.numeric(rownames(urinal_distances)[which.max(closest_distance)])
  } else if(length(possible_urinals) == 0) {
    #if there are no free urinals break the loop and add one to the urinal number in the hypothetical bathroom
    urinal_number <- urinal_number + 1
    break
  }
  #occupy the chosen urinal
  urinals_df$occupied[taken_urinal] <- 1
}
\end{verbatim}

\}

\#if completed \#i.e.~if all users have found a satisfactory free urinal
if(sum(urinals\_df\$occupied, na.rm = TRUE) == n) \{ if(n == 1) \{
\#when n = 1 initial a df to hold the results per n results\_df
\textless{}- data.frame(n = 1, urinals\_required = urinal\_number) \}
else \{ \#otherwise add in a new row to results\_df results\_df
\textless{}- rbind(results\_df, data.frame(n = n, urinals\_required =
urinal\_number)) \} \#increase n to the next number of patrons n
\textless{}- n + 1 \#start with at least n urinals in the next bathroom
\#this is the bare minimum we would need urinal\_number \textless{}- n
\}

\#reintialise the bathroom to see if it is big enough for the n patrons
urinals\_df \textless{}- data.frame(urinal = 1:urinal\_number, occupied
= rep(NA, urinal\_number))

\}

```

We can then plot this. I decided to add a little flair to the plot using
annotate\_custom which is a nice little trick to spice up ggplots

```\{r, warning=FALSE,message=FALSE\} \#load the libraries for plotting
library(ggplot2) library(png) library(grid)

\section{a nice png of a urinal I found
online}\label{a-nice-png-of-a-urinal-i-found-online}

urinal\_image \textless{}- readPNG(``/img/urinal.png'')
\%\textgreater{}\% rasterGrob()

\section{plot the number of urinals needed for any n number of
patrons}\label{plot-the-number-of-urinals-needed-for-any-n-number-of-patrons}

urinals\_plot \textless{}- ggplot(data = results\_df, aes(x = n, y =
urinals\_required)) + geom\_point() + \#mapply a function to paste the
urinal image as an annotation to the graph \#takes the x and y arguments
from the ggplot aesthetic mapply(function(x, y, size) \{
annotation\_custom(urinal\_image, xmin = x - size, ymin = y - size, xmax
= x + size, ymax = y + size) \}, x =
results\_df\(n, y = results_df\)urinals\_required, size = 7) +
\#labelling and etc. ylab(``Urinals Required'') + xlab(``Number of
Patrons'') + ggtitle(``How many urinals are needed for any n number of
socially awkward urinators'', subtitle = ``answer to The Riddler
27/04/2018'') + theme(panel.background = element\_rect(fill =
`lightblue'))

urinals\_plot ```

which gives a surprisingly complex function! I had assume it would be
some simple function of x but clearly something more complex is going
on.

Why this happens become clear if you plot out why M urinals are needed
for N people. Optimally each person would be separated by 1 urinal, but
as the number of urinals increases they become less efficiently packed,
with 2 urinals (neither of which can be used without standing next to
someone) between each urinating person. This eventually reaches a
breaking point and the number of urinals necessary jumps upward.

The formula is known as `The Pay Phone Packing Sequence' (where users of
pay phones don't want to be overheard) and is summarised at
\url{https://oeis.org/A185456}. The formula itself is:

\(f(n) = n + 2 ^ (1 + floor(log(n - 2)))\)

That's all for this weeks riddler.

\href{https://www.youtube.com/watch?v=l3V4KfeJBCQ}{Franz Ferdinand and
Sparks - Piss Off}


\end{document}
