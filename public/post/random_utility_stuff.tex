\documentclass[]{article}
\usepackage{lmodern}
\usepackage{amssymb,amsmath}
\usepackage{ifxetex,ifluatex}
\usepackage{fixltx2e} % provides \textsubscript
\ifnum 0\ifxetex 1\fi\ifluatex 1\fi=0 % if pdftex
  \usepackage[T1]{fontenc}
  \usepackage[utf8]{inputenc}
\else % if luatex or xelatex
  \ifxetex
    \usepackage{mathspec}
  \else
    \usepackage{fontspec}
  \fi
  \defaultfontfeatures{Ligatures=TeX,Scale=MatchLowercase}
\fi
% use upquote if available, for straight quotes in verbatim environments
\IfFileExists{upquote.sty}{\usepackage{upquote}}{}
% use microtype if available
\IfFileExists{microtype.sty}{%
\usepackage{microtype}
\UseMicrotypeSet[protrusion]{basicmath} % disable protrusion for tt fonts
}{}
\usepackage[margin=1in]{geometry}
\usepackage{hyperref}
\hypersetup{unicode=true,
            pdftitle={Untitled},
            pdfborder={0 0 0},
            breaklinks=true}
\urlstyle{same}  % don't use monospace font for urls
\usepackage{color}
\usepackage{fancyvrb}
\newcommand{\VerbBar}{|}
\newcommand{\VERB}{\Verb[commandchars=\\\{\}]}
\DefineVerbatimEnvironment{Highlighting}{Verbatim}{commandchars=\\\{\}}
% Add ',fontsize=\small' for more characters per line
\usepackage{framed}
\definecolor{shadecolor}{RGB}{248,248,248}
\newenvironment{Shaded}{\begin{snugshade}}{\end{snugshade}}
\newcommand{\AlertTok}[1]{\textcolor[rgb]{0.94,0.16,0.16}{#1}}
\newcommand{\AnnotationTok}[1]{\textcolor[rgb]{0.56,0.35,0.01}{\textbf{\textit{#1}}}}
\newcommand{\AttributeTok}[1]{\textcolor[rgb]{0.77,0.63,0.00}{#1}}
\newcommand{\BaseNTok}[1]{\textcolor[rgb]{0.00,0.00,0.81}{#1}}
\newcommand{\BuiltInTok}[1]{#1}
\newcommand{\CharTok}[1]{\textcolor[rgb]{0.31,0.60,0.02}{#1}}
\newcommand{\CommentTok}[1]{\textcolor[rgb]{0.56,0.35,0.01}{\textit{#1}}}
\newcommand{\CommentVarTok}[1]{\textcolor[rgb]{0.56,0.35,0.01}{\textbf{\textit{#1}}}}
\newcommand{\ConstantTok}[1]{\textcolor[rgb]{0.00,0.00,0.00}{#1}}
\newcommand{\ControlFlowTok}[1]{\textcolor[rgb]{0.13,0.29,0.53}{\textbf{#1}}}
\newcommand{\DataTypeTok}[1]{\textcolor[rgb]{0.13,0.29,0.53}{#1}}
\newcommand{\DecValTok}[1]{\textcolor[rgb]{0.00,0.00,0.81}{#1}}
\newcommand{\DocumentationTok}[1]{\textcolor[rgb]{0.56,0.35,0.01}{\textbf{\textit{#1}}}}
\newcommand{\ErrorTok}[1]{\textcolor[rgb]{0.64,0.00,0.00}{\textbf{#1}}}
\newcommand{\ExtensionTok}[1]{#1}
\newcommand{\FloatTok}[1]{\textcolor[rgb]{0.00,0.00,0.81}{#1}}
\newcommand{\FunctionTok}[1]{\textcolor[rgb]{0.00,0.00,0.00}{#1}}
\newcommand{\ImportTok}[1]{#1}
\newcommand{\InformationTok}[1]{\textcolor[rgb]{0.56,0.35,0.01}{\textbf{\textit{#1}}}}
\newcommand{\KeywordTok}[1]{\textcolor[rgb]{0.13,0.29,0.53}{\textbf{#1}}}
\newcommand{\NormalTok}[1]{#1}
\newcommand{\OperatorTok}[1]{\textcolor[rgb]{0.81,0.36,0.00}{\textbf{#1}}}
\newcommand{\OtherTok}[1]{\textcolor[rgb]{0.56,0.35,0.01}{#1}}
\newcommand{\PreprocessorTok}[1]{\textcolor[rgb]{0.56,0.35,0.01}{\textit{#1}}}
\newcommand{\RegionMarkerTok}[1]{#1}
\newcommand{\SpecialCharTok}[1]{\textcolor[rgb]{0.00,0.00,0.00}{#1}}
\newcommand{\SpecialStringTok}[1]{\textcolor[rgb]{0.31,0.60,0.02}{#1}}
\newcommand{\StringTok}[1]{\textcolor[rgb]{0.31,0.60,0.02}{#1}}
\newcommand{\VariableTok}[1]{\textcolor[rgb]{0.00,0.00,0.00}{#1}}
\newcommand{\VerbatimStringTok}[1]{\textcolor[rgb]{0.31,0.60,0.02}{#1}}
\newcommand{\WarningTok}[1]{\textcolor[rgb]{0.56,0.35,0.01}{\textbf{\textit{#1}}}}
\usepackage{graphicx,grffile}
\makeatletter
\def\maxwidth{\ifdim\Gin@nat@width>\linewidth\linewidth\else\Gin@nat@width\fi}
\def\maxheight{\ifdim\Gin@nat@height>\textheight\textheight\else\Gin@nat@height\fi}
\makeatother
% Scale images if necessary, so that they will not overflow the page
% margins by default, and it is still possible to overwrite the defaults
% using explicit options in \includegraphics[width, height, ...]{}
\setkeys{Gin}{width=\maxwidth,height=\maxheight,keepaspectratio}
\IfFileExists{parskip.sty}{%
\usepackage{parskip}
}{% else
\setlength{\parindent}{0pt}
\setlength{\parskip}{6pt plus 2pt minus 1pt}
}
\setlength{\emergencystretch}{3em}  % prevent overfull lines
\providecommand{\tightlist}{%
  \setlength{\itemsep}{0pt}\setlength{\parskip}{0pt}}
\setcounter{secnumdepth}{0}
% Redefines (sub)paragraphs to behave more like sections
\ifx\paragraph\undefined\else
\let\oldparagraph\paragraph
\renewcommand{\paragraph}[1]{\oldparagraph{#1}\mbox{}}
\fi
\ifx\subparagraph\undefined\else
\let\oldsubparagraph\subparagraph
\renewcommand{\subparagraph}[1]{\oldsubparagraph{#1}\mbox{}}
\fi

%%% Use protect on footnotes to avoid problems with footnotes in titles
\let\rmarkdownfootnote\footnote%
\def\footnote{\protect\rmarkdownfootnote}

%%% Change title format to be more compact
\usepackage{titling}

% Create subtitle command for use in maketitle
\newcommand{\subtitle}[1]{
  \posttitle{
    \begin{center}\large#1\end{center}
    }
}

\setlength{\droptitle}{-2em}

  \title{Untitled}
    \pretitle{\vspace{\droptitle}\centering\huge}
  \posttitle{\par}
    \author{}
    \preauthor{}\postauthor{}
    \date{}
    \predate{}\postdate{}
  

\begin{document}
\maketitle

load up tidyverse for some verbose data munging

\begin{Shaded}
\begin{Highlighting}[]
\KeywordTok{library}\NormalTok{(tidyverse)}
\end{Highlighting}
\end{Shaded}

load in the data (Vicer bundle choice data from March)

\begin{Shaded}
\begin{Highlighting}[]
\NormalTok{data <-}\StringTok{ }\KeywordTok{dir}\NormalTok{(}\StringTok{"."}\NormalTok{) }\OperatorTok
\StringTok{  }\CommentTok{#load in all .csvs}
\StringTok{  }\CommentTok{#trial by trial results}
\StringTok{  }\NormalTok{.[}\KeywordTok{grepl}\NormalTok{(}\StringTok{"}\CharTok{\textbackslash{}\textbackslash{}}\StringTok{.csv$"}\NormalTok{, .)] }\OperatorTok
\StringTok{  }\KeywordTok{file.path}\NormalTok{(}\StringTok{"."}\NormalTok{, .) }\OperatorTok
\StringTok{  }\KeywordTok{lapply}\NormalTok{(., read.csv, }\DataTypeTok{stringsAsFactors =} \OtherTok{FALSE}\NormalTok{) }\OperatorTok
\StringTok{  }\CommentTok{#bind data together}
\StringTok{  }\KeywordTok{map_df}\NormalTok{(I)}

\KeywordTok{table}\NormalTok{(data}\OperatorTok{$}\NormalTok{date)}
\end{Highlighting}
\end{Shaded}

We have data from 3 separate days, but I'm just going to pool them
together for larger n

Plot the data in the standard bundle choice logistic regression format

\begin{Shaded}
\begin{Highlighting}[]
\NormalTok{bcb_data <-}\StringTok{ }\NormalTok{data }\OperatorTok
\StringTok{  }\CommentTok{#only want data pertaining to bundle choice task}
\StringTok{  }\KeywordTok{filter}\NormalTok{(subtask }\OperatorTok{==}\StringTok{ "bundle_choice"}\NormalTok{) }\OperatorTok
\StringTok{  }\CommentTok{#select only interesting variables}
\StringTok{  }\KeywordTok{mutate}\NormalTok{(}\DataTypeTok{bundle_water_ml =}\NormalTok{ second_budget_value }\OperatorTok{*}\StringTok{ }\NormalTok{budget_magnitude,}
         \DataTypeTok{non_bundle_water_ml =}\NormalTok{ budget_value }\OperatorTok{*}\StringTok{ }\NormalTok{budget_magnitude) }\OperatorTok
\StringTok{  }\KeywordTok{select}\NormalTok{(bundle_water_ml, non_bundle_water_ml, }\DataTypeTok{juice_ml =}\NormalTok{ reward_magnitude, results) }\OperatorTok
\StringTok{  }\CommentTok{#binary yes/no for choosing the bundle or not}
\StringTok{  }\KeywordTok{mutate}\NormalTok{(}\DataTypeTok{bundle_chosen =} \KeywordTok{case_when}\NormalTok{(}
    \KeywordTok{grepl}\NormalTok{(}\StringTok{"fractal_chosen"}\NormalTok{, results) }\OperatorTok{~}\StringTok{ }\DecValTok{1}\NormalTok{,}
    \KeywordTok{grepl}\NormalTok{(}\StringTok{"budget_chosen"}\NormalTok{, results) }\OperatorTok{~}\StringTok{ }\DecValTok{0}
\NormalTok{  )) }\OperatorTok
\StringTok{  }\CommentTok{#remove trials where neither is chosen}
\StringTok{  }\CommentTok{#i.e. errors}
\StringTok{  }\KeywordTok{filter}\NormalTok{(}\OperatorTok{!}\KeywordTok{is.na}\NormalTok{(bundle_chosen))}
  
\CommentTok{#nice quick function for plotting a logistic regression curve}
\NormalTok{binomial_smooth <-}\StringTok{ }\ControlFlowTok{function}\NormalTok{(...) \{}
  \KeywordTok{geom_smooth}\NormalTok{(}\DataTypeTok{method =} \StringTok{"glm"}\NormalTok{, }\DataTypeTok{method.args =} \KeywordTok{list}\NormalTok{(}\DataTypeTok{family =} \StringTok{"binomial"}\NormalTok{), ...)}
\NormalTok{\}}

\NormalTok{p <-}\StringTok{ }\NormalTok{bcb_data }\OperatorTok
\StringTok{  }\CommentTok{#group data by bundle combinations of reward and water}
\StringTok{  }\KeywordTok{group_by}\NormalTok{(juice_ml, bundle_water_ml) }\OperatorTok
\StringTok{  }\CommentTok{#what fraction of the time is the bundle chosen}
\StringTok{  }\KeywordTok{summarise}\NormalTok{(}\DataTypeTok{fraction_bundle_choice =} \KeywordTok{mean}\NormalTok{(bundle_chosen)) }\OperatorTok
\StringTok{  }\CommentTok{#pipe into a plot}
\StringTok{  }\KeywordTok{ggplot}\NormalTok{(., }\KeywordTok{aes}\NormalTok{(}\DataTypeTok{x =}\NormalTok{ bundle_water_ml, }\DataTypeTok{y =}\NormalTok{ fraction_bundle_choice, }\DataTypeTok{colour =} \KeywordTok{factor}\NormalTok{(juice_ml))) }\OperatorTok{+}
\StringTok{  }\CommentTok{#plot the likelihood of choosing the bundle for each juice/water combination}
\StringTok{  }\KeywordTok{geom_point}\NormalTok{() }\OperatorTok{+}
\StringTok{  }\CommentTok{#add in the logistic fit line}
\StringTok{  }\CommentTok{#n.b. se is calculated per point so is massive}
\StringTok{  }\KeywordTok{binomial_smooth}\NormalTok{(}\DataTypeTok{se =} \OtherTok{FALSE}\NormalTok{) }\OperatorTok{+}
\StringTok{  }\CommentTok{#add in the colour scale}
\StringTok{  }\KeywordTok{scale_colour_manual}\NormalTok{(}\DataTypeTok{values =} \KeywordTok{c}\NormalTok{(}\StringTok{"red"}\NormalTok{, }\StringTok{"blue"}\NormalTok{, }\StringTok{"green"}\NormalTok{), }
                      \DataTypeTok{name =} \StringTok{"reward juice (/ml)"}\NormalTok{) }\OperatorTok{+}
\StringTok{  }\CommentTok{#add in some labels}
\StringTok{  }\KeywordTok{labs}\NormalTok{(}\DataTypeTok{title =} \StringTok{"Bundle Choice Logistic Regression"}\NormalTok{,}
       \DataTypeTok{subtitle =} \StringTok{"data from Vicer on bundle choice task in March"}\NormalTok{,}
       \DataTypeTok{x =} \StringTok{"bundle water (/ml)"}\NormalTok{,}
       \DataTypeTok{y =} \StringTok{"% chooses bundle"}\NormalTok{) }\OperatorTok{+}
\StringTok{  }\CommentTok{#remove extra stuff}
\StringTok{  }\KeywordTok{theme_minimal}\NormalTok{()}

\CommentTok{#plot it}
\NormalTok{p}
\end{Highlighting}
\end{Shaded}

\[p(A)=\frac{1}{e^{-\phi(u(A) - u(B))}\]


\end{document}
