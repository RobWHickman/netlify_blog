\documentclass[]{article}
\usepackage{lmodern}
\usepackage{amssymb,amsmath}
\usepackage{ifxetex,ifluatex}
\usepackage{fixltx2e} % provides \textsubscript
\ifnum 0\ifxetex 1\fi\ifluatex 1\fi=0 % if pdftex
  \usepackage[T1]{fontenc}
  \usepackage[utf8]{inputenc}
\else % if luatex or xelatex
  \ifxetex
    \usepackage{mathspec}
  \else
    \usepackage{fontspec}
  \fi
  \defaultfontfeatures{Ligatures=TeX,Scale=MatchLowercase}
\fi
% use upquote if available, for straight quotes in verbatim environments
\IfFileExists{upquote.sty}{\usepackage{upquote}}{}
% use microtype if available
\IfFileExists{microtype.sty}{%
\usepackage{microtype}
\UseMicrotypeSet[protrusion]{basicmath} % disable protrusion for tt fonts
}{}
\usepackage[margin=1in]{geometry}
\usepackage{hyperref}
\hypersetup{unicode=true,
            pdfborder={0 0 0},
            breaklinks=true}
\urlstyle{same}  % don't use monospace font for urls
\usepackage{color}
\usepackage{fancyvrb}
\newcommand{\VerbBar}{|}
\newcommand{\VERB}{\Verb[commandchars=\\\{\}]}
\DefineVerbatimEnvironment{Highlighting}{Verbatim}{commandchars=\\\{\}}
% Add ',fontsize=\small' for more characters per line
\usepackage{framed}
\definecolor{shadecolor}{RGB}{248,248,248}
\newenvironment{Shaded}{\begin{snugshade}}{\end{snugshade}}
\newcommand{\AlertTok}[1]{\textcolor[rgb]{0.94,0.16,0.16}{#1}}
\newcommand{\AnnotationTok}[1]{\textcolor[rgb]{0.56,0.35,0.01}{\textbf{\textit{#1}}}}
\newcommand{\AttributeTok}[1]{\textcolor[rgb]{0.77,0.63,0.00}{#1}}
\newcommand{\BaseNTok}[1]{\textcolor[rgb]{0.00,0.00,0.81}{#1}}
\newcommand{\BuiltInTok}[1]{#1}
\newcommand{\CharTok}[1]{\textcolor[rgb]{0.31,0.60,0.02}{#1}}
\newcommand{\CommentTok}[1]{\textcolor[rgb]{0.56,0.35,0.01}{\textit{#1}}}
\newcommand{\CommentVarTok}[1]{\textcolor[rgb]{0.56,0.35,0.01}{\textbf{\textit{#1}}}}
\newcommand{\ConstantTok}[1]{\textcolor[rgb]{0.00,0.00,0.00}{#1}}
\newcommand{\ControlFlowTok}[1]{\textcolor[rgb]{0.13,0.29,0.53}{\textbf{#1}}}
\newcommand{\DataTypeTok}[1]{\textcolor[rgb]{0.13,0.29,0.53}{#1}}
\newcommand{\DecValTok}[1]{\textcolor[rgb]{0.00,0.00,0.81}{#1}}
\newcommand{\DocumentationTok}[1]{\textcolor[rgb]{0.56,0.35,0.01}{\textbf{\textit{#1}}}}
\newcommand{\ErrorTok}[1]{\textcolor[rgb]{0.64,0.00,0.00}{\textbf{#1}}}
\newcommand{\ExtensionTok}[1]{#1}
\newcommand{\FloatTok}[1]{\textcolor[rgb]{0.00,0.00,0.81}{#1}}
\newcommand{\FunctionTok}[1]{\textcolor[rgb]{0.00,0.00,0.00}{#1}}
\newcommand{\ImportTok}[1]{#1}
\newcommand{\InformationTok}[1]{\textcolor[rgb]{0.56,0.35,0.01}{\textbf{\textit{#1}}}}
\newcommand{\KeywordTok}[1]{\textcolor[rgb]{0.13,0.29,0.53}{\textbf{#1}}}
\newcommand{\NormalTok}[1]{#1}
\newcommand{\OperatorTok}[1]{\textcolor[rgb]{0.81,0.36,0.00}{\textbf{#1}}}
\newcommand{\OtherTok}[1]{\textcolor[rgb]{0.56,0.35,0.01}{#1}}
\newcommand{\PreprocessorTok}[1]{\textcolor[rgb]{0.56,0.35,0.01}{\textit{#1}}}
\newcommand{\RegionMarkerTok}[1]{#1}
\newcommand{\SpecialCharTok}[1]{\textcolor[rgb]{0.00,0.00,0.00}{#1}}
\newcommand{\SpecialStringTok}[1]{\textcolor[rgb]{0.31,0.60,0.02}{#1}}
\newcommand{\StringTok}[1]{\textcolor[rgb]{0.31,0.60,0.02}{#1}}
\newcommand{\VariableTok}[1]{\textcolor[rgb]{0.00,0.00,0.00}{#1}}
\newcommand{\VerbatimStringTok}[1]{\textcolor[rgb]{0.31,0.60,0.02}{#1}}
\newcommand{\WarningTok}[1]{\textcolor[rgb]{0.56,0.35,0.01}{\textbf{\textit{#1}}}}
\usepackage{graphicx,grffile}
\makeatletter
\def\maxwidth{\ifdim\Gin@nat@width>\linewidth\linewidth\else\Gin@nat@width\fi}
\def\maxheight{\ifdim\Gin@nat@height>\textheight\textheight\else\Gin@nat@height\fi}
\makeatother
% Scale images if necessary, so that they will not overflow the page
% margins by default, and it is still possible to overwrite the defaults
% using explicit options in \includegraphics[width, height, ...]{}
\setkeys{Gin}{width=\maxwidth,height=\maxheight,keepaspectratio}
\IfFileExists{parskip.sty}{%
\usepackage{parskip}
}{% else
\setlength{\parindent}{0pt}
\setlength{\parskip}{6pt plus 2pt minus 1pt}
}
\setlength{\emergencystretch}{3em}  % prevent overfull lines
\providecommand{\tightlist}{%
  \setlength{\itemsep}{0pt}\setlength{\parskip}{0pt}}
\setcounter{secnumdepth}{0}
% Redefines (sub)paragraphs to behave more like sections
\ifx\paragraph\undefined\else
\let\oldparagraph\paragraph
\renewcommand{\paragraph}[1]{\oldparagraph{#1}\mbox{}}
\fi
\ifx\subparagraph\undefined\else
\let\oldsubparagraph\subparagraph
\renewcommand{\subparagraph}[1]{\oldsubparagraph{#1}\mbox{}}
\fi

%%% Use protect on footnotes to avoid problems with footnotes in titles
\let\rmarkdownfootnote\footnote%
\def\footnote{\protect\rmarkdownfootnote}

%%% Change title format to be more compact
\usepackage{titling}

% Create subtitle command for use in maketitle
\providecommand{\subtitle}[1]{
  \posttitle{
    \begin{center}\large#1\end{center}
    }
}

\setlength{\droptitle}{-2em}

  \title{}
    \pretitle{\vspace{\droptitle}}
  \posttitle{}
    \author{}
    \preauthor{}\postauthor{}
    \date{}
    \predate{}\postdate{}
  

\begin{document}

title: ``The Guardian Knowledge July 2019'' author: ``Robert Hickman''
date: `2019-07-07' output: html\_document: df\_print: paged header:
caption: '' image: '' slug: guardian\_knowledge\_july tags: - rstats -
football - the\_knowledge categories: {[}{]} ---

\begin{Shaded}
\begin{Highlighting}[]
\KeywordTok{library}\NormalTok{(tidyverse)}
\KeywordTok{library}\NormalTok{(broom)}
\KeywordTok{library}\NormalTok{(engsoccerdata)}
\end{Highlighting}
\end{Shaded}

\begin{Shaded}
\begin{Highlighting}[]
\NormalTok{league_data <-}\StringTok{ }\NormalTok{engsoccerdata}\OperatorTok{::}\NormalTok{england }\OperatorTok
\StringTok{  }\KeywordTok{select}\NormalTok{(}\DataTypeTok{season =}\NormalTok{ Season, division, home, visitor, hgoal, vgoal) }\OperatorTok
\StringTok{  }\KeywordTok{gather}\NormalTok{(}\StringTok{"location"}\NormalTok{, }\StringTok{"team"}\NormalTok{, }\OperatorTok{-}\NormalTok{season, }\OperatorTok{-}\NormalTok{division, }\OperatorTok{-}\NormalTok{hgoal, }\OperatorTok{-}\NormalTok{vgoal) }\OperatorTok
\StringTok{  }\KeywordTok{mutate}\NormalTok{(}
    \DataTypeTok{g_for =} \KeywordTok{case_when}\NormalTok{(}
\NormalTok{      location }\OperatorTok{==}\StringTok{ "home"} \OperatorTok{~}\StringTok{ }\NormalTok{hgoal,}
\NormalTok{      location }\OperatorTok{==}\StringTok{ "visitor"} \OperatorTok{~}\StringTok{ }\NormalTok{vgoal}
\NormalTok{    ),}
    \DataTypeTok{g_ag =} \KeywordTok{case_when}\NormalTok{(}
\NormalTok{      location }\OperatorTok{==}\StringTok{ "home"} \OperatorTok{~}\StringTok{ }\NormalTok{vgoal,}
\NormalTok{      location }\OperatorTok{==}\StringTok{ "visitor"} \OperatorTok{~}\StringTok{ }\NormalTok{hgoal}
\NormalTok{    )) }\OperatorTok
\StringTok{  }\KeywordTok{mutate}\NormalTok{(}
    \DataTypeTok{points =} \KeywordTok{case_when}\NormalTok{(}
\NormalTok{      g_for }\OperatorTok{>}\StringTok{ }\NormalTok{g_ag }\OperatorTok{&}\StringTok{ }\NormalTok{season }\OperatorTok{<}\StringTok{ }\DecValTok{1981} \OperatorTok{~}\StringTok{ }\DecValTok{2}\NormalTok{,}
\NormalTok{      g_for }\OperatorTok{>}\StringTok{ }\NormalTok{g_ag }\OperatorTok{&}\StringTok{ }\NormalTok{season }\OperatorTok{>}\StringTok{ }\DecValTok{1980} \OperatorTok{~}\StringTok{ }\DecValTok{3}\NormalTok{,}
\NormalTok{      g_for }\OperatorTok{==}\StringTok{ }\NormalTok{g_ag }\OperatorTok{~}\StringTok{ }\DecValTok{1}\NormalTok{,}
\NormalTok{      g_for }\OperatorTok{<}\StringTok{ }\NormalTok{g_ag }\OperatorTok{~}\StringTok{ }\DecValTok{0}
\NormalTok{    ),}
    \DataTypeTok{gd =}\NormalTok{ g_for }\OperatorTok{-}\StringTok{ }\NormalTok{g_ag}
\NormalTok{  ) }\OperatorTok
\StringTok{  }\KeywordTok{group_by}\NormalTok{(season, division, team) }\OperatorTok
\StringTok{  }\KeywordTok{summarise}\NormalTok{(}\DataTypeTok{points =} \KeywordTok{sum}\NormalTok{(points),}
            \DataTypeTok{gd =} \KeywordTok{sum}\NormalTok{(gd),}
            \DataTypeTok{g_for =} \KeywordTok{sum}\NormalTok{(g_for)) }\OperatorTok
\StringTok{  }\KeywordTok{arrange}\NormalTok{(}\OperatorTok{-}\NormalTok{points, }\OperatorTok{-}\NormalTok{gd, }\OperatorTok{-}\NormalTok{g_for) }\OperatorTok
\StringTok{  }\KeywordTok{mutate}\NormalTok{(}\DataTypeTok{league_pos =} \KeywordTok{rank}\NormalTok{(}\OperatorTok{-}\NormalTok{points, }\DataTypeTok{ties.method =} \StringTok{"first"}\NormalTok{),}
         \DataTypeTok{alph_order =} \KeywordTok{rank}\NormalTok{(team, }\DataTypeTok{ties.method =} \StringTok{"first"}\NormalTok{)) }\OperatorTok
\StringTok{  }\KeywordTok{select}\NormalTok{(season, division, team, league_pos, alph_order) }\OperatorTok
\StringTok{  }\KeywordTok{split}\NormalTok{(., }\DataTypeTok{f =} \KeywordTok{list}\NormalTok{(.}\OperatorTok{$}\NormalTok{season, .}\OperatorTok{$}\NormalTok{division)) }\OperatorTok
\StringTok{  }\KeywordTok{keep}\NormalTok{(}\ControlFlowTok{function}\NormalTok{(x) }\KeywordTok{nrow}\NormalTok{(x) }\OperatorTok{>}\StringTok{ }\DecValTok{0}\NormalTok{)}
\end{Highlighting}
\end{Shaded}

\begin{Shaded}
\begin{Highlighting}[]
\NormalTok{get_spearman_rank <-}\StringTok{ }\ControlFlowTok{function}\NormalTok{(league_position, alphabetical_order, }\DataTypeTok{stat_method =} \StringTok{"spearman"}\NormalTok{) \{}
\NormalTok{  test <-}\StringTok{ }\KeywordTok{cor.test}\NormalTok{(league_position, alphabetical_order, }\DataTypeTok{method =}\NormalTok{ stat_method) }\OperatorTok
\StringTok{    }\KeywordTok{tidy}\NormalTok{()}
\NormalTok{\}}
\end{Highlighting}
\end{Shaded}

\begin{Shaded}
\begin{Highlighting}[]
\NormalTok{correlations <-}\StringTok{ }\NormalTok{league_data }\OperatorTok
\StringTok{  }\KeywordTok{map_df}\NormalTok{(., }\ControlFlowTok{function}\NormalTok{(data) \{}
    \KeywordTok{cor.test}\NormalTok{(}
\NormalTok{      data}\OperatorTok{$}\NormalTok{league_pos,}
\NormalTok{      data}\OperatorTok{$}\NormalTok{alph_order,}
      \DataTypeTok{method =} \StringTok{"spearman"}
\NormalTok{    ) }\OperatorTok
\StringTok{      }\KeywordTok{tidy}\NormalTok{() }\OperatorTok
\StringTok{      }\KeywordTok{mutate}\NormalTok{(}\DataTypeTok{season =} \KeywordTok{unique}\NormalTok{(data}\OperatorTok{$}\NormalTok{season),}
             \DataTypeTok{division =} \KeywordTok{unique}\NormalTok{(data}\OperatorTok{$}\NormalTok{division))}
\NormalTok{  \}) }\OperatorTok
\StringTok{  }\KeywordTok{filter}\NormalTok{(p.value }\OperatorTok{<}\StringTok{ }\FloatTok{0.05}\NormalTok{)}
\end{Highlighting}
\end{Shaded}

Let's imagine a very small league (say 8 teams).

\begin{Shaded}
\begin{Highlighting}[]
\NormalTok{first_letter_names <-}\StringTok{ }\NormalTok{league_data }\OperatorTok
\StringTok{  }\KeywordTok{bind_rows}\NormalTok{() }\OperatorTok
\StringTok{  }\KeywordTok{ungroup}\NormalTok{() }\OperatorTok
\StringTok{  }\KeywordTok{mutate}\NormalTok{(}\DataTypeTok{first_letter =} \KeywordTok{gsub}\NormalTok{(}\StringTok{"(^.)(.*)"}\NormalTok{, }\StringTok{"}\CharTok{\textbackslash{}\textbackslash{}}\StringTok{1"}\NormalTok{, team)) }\OperatorTok
\StringTok{  }\KeywordTok{filter}\NormalTok{(season }\OperatorTok{>}\StringTok{ }\DecValTok{1992} \OperatorTok{&}
\StringTok{           }\NormalTok{division }\OperatorTok{==}\StringTok{ }\DecValTok{1} \OperatorTok{&}
\StringTok{           }\NormalTok{first_letter }\OperatorTok\StringTok{ }\KeywordTok{toupper}\NormalTok{(letters[}\DecValTok{1}\OperatorTok{:}\DecValTok{6}\NormalTok{])}
\NormalTok{         ) }\OperatorTok
\StringTok{  }\KeywordTok{filter}\NormalTok{(}\OperatorTok{!}\KeywordTok{duplicated}\NormalTok{(first_letter)) }\OperatorTok
\StringTok{  }\KeywordTok{select}\NormalTok{(team) }\OperatorTok
\StringTok{  }\KeywordTok{arrange}\NormalTok{(team) }\OperatorTok
\StringTok{  }\KeywordTok{print}\NormalTok{()}
\end{Highlighting}
\end{Shaded}

\begin{verbatim}
## # A tibble: 6 x 1
##   team            
##   <chr>           
## 1 Arsenal         
## 2 Blackburn Rovers
## 3 Coventry City   
## 4 Derby County    
## 5 Everton         
## 6 Fulham
\end{verbatim}

So for the league to finish in alphabetical order, we first need the
team that is first alphabetically (Arsenal) to finish in first position.
Assuming all teams have an equal chance of winning the league, the
chance of this is obviously

\[ p(Arsenal = 1) =  \frac{1}{n}\]

Then we need the second team (Blackburn Rovers), to finish in second.
This is predicated on Arsenal already finishing in first position, so
the chance becomes

\[ p(Blackburn = 2 | Arsenal = 1) = \frac{1}{n-1} \]

and so on until the last team (Fulham) just have to slot into the only
position left (n, 6th in our example)

Thus the total chance becomes

\[ \frac{1}{n} \cdot \frac{1}{n-1} ... \cdot \frac{1}{1} \]

which can also be written

\[ p(ordered) = \prod_{n = 1}^{N} \frac{1}{n}\]

which multiplies out to

\[ p(ordered) = \frac{1}{n!} \]

so for our very small league the chance of n (assumed equally strong
teams)

\begin{Shaded}
\begin{Highlighting}[]
\KeywordTok{factorial}\NormalTok{(}\KeywordTok{nrow}\NormalTok{(first_letter_names))}
\end{Highlighting}
\end{Shaded}

\begin{verbatim}
## [1] 720
\end{verbatim}


\end{document}
